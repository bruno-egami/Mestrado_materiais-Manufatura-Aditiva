\documentclass{beamer}

% --- Configuração do Tema e Pacotes ---
\usetheme{MU} 
% Ajuste do rodapé e numeração
\setbeamertemplate{footline}{%
    \ifnum\insertframenumber=1
        % Slide 1: rodapé com data e número
        \leavevmode\hbox{\begin{beamercolorbox}[wd=\paperwidth,ht=2.5ex,dp=1.125ex,left]{date in head/foot}%
            \hspace{1em}\insertshortdate\hfill\insertframenumber/\inserttotalframenumber\hspace{1em}
        \end{beamercolorbox}}
    \else
        % Slides 2+: pequeno rodapé visível próximo à margem inferior, sem número
        \leavevmode\hbox{\begin{beamercolorbox}[wd=\paperwidth,ht=1.2ex,dp=1ex,center]{date in head/foot}%
            % thin footer bar (keeps visual separation near bottom)
        \end{beamercolorbox}}
    \fi
}

% Número do slide no canto superior direito
\setbeamertemplate{headline}{%
            \ifnum\insertframenumber>1
                % Número do slide apenas no canto superior direito
                \leavevmode\hbox{\begin{beamercolorbox}[wd=\paperwidth,ht=2.5ex,dp=1.125ex,right]{section in head/foot}%
                    \hfill\insertframenumber/\inserttotalframenumber\hspace{1em}
                \end{beamercolorbox}}
            \fi
}
\usepackage[utf8]{inputenc}
\usepackage[T1]{fontenc}
\usepackage[brazil]{babel}
\usepackage{amsmath}
\usepackage{amsfonts}
\usepackage{amssymb}
\usepackage{capt-of}
\usepackage{graphicx} 
%\usepackage{caption} % removido: não é necessário em beamer e causou usos desnecessários

% --- Configuração de Fundo ---
% Nota: o background global foi removido para evitar que a imagem seja
% replicada e escalada em todos os slides. A imagem será aplicada apenas
% na capa (slide 1) abaixo, na sua dimensão original.

% Dados da Apresentação
\title[\tiny DIW de Cerâmicas Avançadas]{\Large Manufatura Aditiva Livre de Solventes e Ligantes de Cerâmicas Derivadas de Polímeros: Otimização Reológica e Desempenho Estrutural}
\subtitle{\normalsize Revisão do Artigo: Viswanadha et al., *Additive Manufacturing*, 2025}

\author{Acad. Bruno Kenji Nishitani Egami\\
        Orientador: Profº Dr. Douglas A. Simon}
\institute{Manufatura Aditiva - PPGTEM 2025/02\\
           IFRS Campus Farroupilha}
\date{\today}

\begin{document}

% --- Capa ---
% Aplicar background somente nesta capa, usando a imagem no tamanho original.
% Usamos \setbeamertemplate localmente e em seguida limpamos o template.
% Criar uma capa customizada: texto à esquerda, imagem (1/3 largura)
% à direita. Evita usar o background canvas que pode interferir no
% layout da titlepage do tema.
\begin{frame}[plain]
    % Fundo branco para garantir contraste
    \setbeamercolor{background canvas}{bg=white}
    % Logotipo no canto superior esquerdo, ajustado mais abaixo
    \vspace{1cm}
    \raisebox{0pt}[0pt][0pt]{\makebox[0pt][l]{\includegraphics[width=3.5cm,keepaspectratio]{background_16_9.png}}}
    % Texto centralizado ocupando toda a largura
    \vspace{2cm}
    \begin{center}
        {\color{black}\usebeamerfont{title}\inserttitle\par}
        \vspace{0.5cm}
        {\color{black}\usebeamerfont{subtitle}\insertsubtitle\par}
        \vspace{1cm}
        {\color{black}\usebeamerfont{author} \insertauthor\par}
        \vspace{0.3cm}
        {\color{black}\usebeamerfont{institute}\insertinstitute\par}
        \vspace{0.3cm}
        {\color{black}\usebeamerfont{date} \insertdate\par}
    \end{center}
\end{frame}

% --- Sumário ---
\begin{frame}{Roteiro da Apresentação}
    \tableofcontents
\end{frame}

% =================================================================
% SEÇÃO 1: INTRODUÇÃO E INOVAÇÃO
% =================================================================
\section{Inovação: A Formulação Binder-Free}

\begin{frame}{O Desafio da Manufatura Aditiva de Cerâmicas}
    \begin{itemize}
    \item As Cerâmicas de Carbeto de Silício (SiC) são críticas para aplicações aeroespaciais devido à alta resistência e resiliência térmica.
        \item O processo DIW (Direct Ink Writing) é promissor, mas a maioria das tintas depende de:
        \begin{itemize}
            \item \textbf{Ligantes (Binders):} Exigem *de-binding* térmico, causando retração irregular e rachaduras.
            \item \textbf{Solventes:} Voláteis, introduzem toxicidade, e sua evaporação induz instabilidades estruturais (empenamento, microfissuras e poros).
        \end{itemize}
    \end{itemize}
\end{frame}

\begin{frame}{Inovação Central do Artigo}
    % Texto ocupando a parte superior (largura total)
    \vspace{-2mm}
    \begin{minipage}[t]{\textwidth}
        \begin{itemize}
            \item \textbf{Abordagem Livre de Solventes e Ligantes (\textit{Solvent- and Binder-free})} [2, 5].
            \item \textbf{Material Chave:} Utilização do \textbf{policarbosilano SMP-10} [5].
            \begin{itemize}
                \item O SMP-10 atua como o \textbf{precursor cerâmico} e a \textbf{fase líquida} para ligar as nanopartículas de $\beta$-SiC [2, 5].
                \item Simplifica o pós-processamento, minimizando defeitos estruturais [4, 5].
            \end{itemize}
        \end{itemize}
    \end{minipage}

    % Forçar a figura para a região inferior do slide
    \vfill
    \begin{center}
        \includegraphics[width=0.75\textwidth,keepaspectratio]{esquema_processo.png}\\[0.6ex]
        \captionof{figure}{\footnotesize Esquema do processo DIW livre de ligantes e solventes.}
    \end{center}
\end{frame}

% =================================================================
% SEÇÃO 2: REOLOGIA E PRINTABILIDADE
% =================================================================
\section{Reologia e Estabilidade da Tinta}

\begin{frame}{Otimização Reológica para o DIW}
    \begin{columns}
        \begin{column}{0.5\textwidth}
            \begin{itemize}
                \item A tinta ideal deve ser \textbf{\textit{shear-thinning}} para extrusão suave [6].
                \item Deve ter elasticidade suficiente para \textbf{reter a forma} após a deposição [6, 7].
            \end{itemize}
            	extbf{Critério de Printabilidade (Ponto de Gel):}
            \begin{itemize}
                \item O critério crucial é o \textbf{Ponto de Gel} ($G' = G''$) ser \textbf{superior a $1000 \text{ Pa}$} [8, 9].
                \item \textbf{Resultado:} A tinta com \textbf{60 wt\% SiC} mostrou excelente retenção, superando o limite [9, 10].
            \end{itemize}
            	extbf{Modelos Teóricos:} Delimitam os limites inferior ($\approx 6.62 \text{ mm}$) e superior ($\approx 14 \text{ mm}$) [18, 19].
        \end{column}
        \begin{column}{0.5\textwidth}
            \centering
            \includegraphics[width=\columnwidth,height=0.40\textheight,keepaspectratio]{curva_reologia.png}
            \captionof{figure}{\footnotesize Curvas de Módulo de Armazenamento ($G'$) e Módulo de Perda ($G''$) em função do *stress* (Ponto de Gel).}
        \end{column}
    \end{columns}
\end{frame}

\begin{frame}{Limites de Estabilidade Estrutural ($h_{max}$)}
    \begin{columns}
        \begin{column}{0.5\textwidth}
            \begin{itemize}
                \item O artigo investigou a \textbf{Altura Máxima Imprimível ($h_{max}$)} [11, 12].
                \item \textbf{Suporte Lateral ($d_w$):} Maior espaçamento levou a $h_{max}$ reduzida ($\approx 7 \text{ mm}$). O espaçamento reduzido é crucial para estruturas mais altas [14, 15].
                \item \textbf{Largura da Camada Base ($l_w$):} Camadas base mais largas aumentaram a $h_{max}$ ($\approx 20 \text{ mm}$) [2, 16].
            \end{itemize}
        \end{column}
        \begin{column}{0.5\textwidth}
            \centering
            \includegraphics[width=\columnwidth,height=0.32\textheight,keepaspectratio]{estabilidade_impressao.png}
            \captionof{figure}{\footnotesize Imagens das estruturas impressas (SEM) e gráfico da altura máxima ($h_{max}$) versus espaçamento de parede ($d_w$).}
        \end{column}
    \end{columns}
\end{frame}

% =================================================================
% SEÇÃO 3: RESULTADOS PÓS-PROCESSAMENTO
% =================================================================
\section{Desempenho Térmico e Mecânico}

\begin{frame}{Transformação de Fase e Retração}
    \begin{columns}
        \begin{column}{0.5\textwidth}
            \begin{itemize}
                    \item \textbf{Pós-Processamento:} Sinterizadas a $1500^\circ\text{C}$ e $2300^\circ\text{C}$ [20].
                    \item \textbf{Fases (XRD):} $\beta$-SiC predominante a $1500^\circ\text{C}$. Transformação para $\alpha$-SiC a $2300^\circ\text{C}$ [21].
                    \item \textbf{Retração:} A $2300^\circ\text{C}$, a retração é maior e \textbf{anisotrópica}, com alta retração fora do plano ($\approx 17\%$) [15, 22].
            \end{itemize}
        \end{column}
        \begin{column}{0.5\textwidth}
            \centering
                    \includegraphics[width=\columnwidth,height=0.44\textheight,keepaspectratio]{xrd_retracao.png}
            \captionof{figure}{\footnotesize Padrões de Difração de Raios X (XRD) comparando as fases a $1500^\circ\text{C}$ e $2300^\circ\text{C}$.}
        \end{column}
    \end{columns}
\end{frame}

\begin{frame}{Propriedades Mecânicas e Térmicas}
    \begin{columns}
        \begin{column}{0.5\textwidth}
            \begin{itemize}
                \item \textbf{Resistência Mecânica (Compressão):}
                \begin{itemize}
                    \item Amostras sinterizadas a $1500^\circ\text{C}$ mostraram maior resistência (média $9.62\ \text{MPa}$) em comparação com $2300^\circ\text{C}$.
                    \item A maior retração a $2300^\circ\text{C}$ explica parte da perda de resistência.
                \end{itemize}
                \item \textbf{Isolamento Térmico:} Microestrutura mais porosa a $1500^\circ\text{C}$ melhora o isolamento térmico.
            \end{itemize}
        \end{column}
        \begin{column}{0.5\textwidth}
            \centering
            \includegraphics[height=0.70\textheight,keepaspectratio]{resultados_mecanicos_termicos-vertical.png}
        \captionof{figure}{\footnotesize Comparação da resistência à compressão e do desempenho de isolamento térmico.}
        \end{column}
    \end{columns}
\end{frame}

% =================================================================
% SEÇÃO 4: CONCLUSÃO
% =================================================================
\section{Conclusão}

\begin{frame}{Conclusões Chave}
    \begin{itemize}
        \item O método \textbf{DIW livre de solventes e ligantes} é uma abordagem conveniente e eficiente para cerâmicas avançadas [2, 15].
        \item A \textbf{reologia} é o fator limitante. A printabilidade ideal requer um \textbf{Ponto de Gel $> 1000 \text{ Pa}$} [8, 25].
        \item A \textbf{estabilidade estrutural} é altamente dependente da geometria [2, 15, 26].
        \item \textbf{Resultado Surpreendente:} O tratamento térmico a $1500^\circ\text{C}$ forneceu \textbf{melhores propriedades mecânicas e isolamento térmico} do que a $2300^\circ\text{C}$ [23, 24].
    \end{itemize}
\end{frame}

\begin{frame}{Relevância e Perspectivas Futuras}
    \begin{itemize}
        \item Este trabalho fornece a fundação para equilibrar printabilidade, isolamento térmico e desempenho mecânico [15].
        \item O uso do DIW com formulações livres de solventes aprimora o controle sobre a evolução microestrutural [2, 15].
    \end{itemize}
\end{frame}

% =================================================================
% SEÇÃO 5: REFERÊNCIAS E FIM
% =================================================================
\section{Referências}

\begin{frame}{Referências}
    
    \begin{itemize}
        \item VISWANADHA, P. et al. Solvent- and binder-free additive manufacturing of polymer-derived ceramics: Rheological optimization and structural performance. \textit{Additive Manufacturing}, v. 84, p. 104278, 2025.
    \end{itemize}
    
\end{frame}

% ÚLTIMO SLIDE: AGRADECIMENTO (apenas o texto)
\begin{frame}
    \vfill
    \centering
    \textbf{\Huge Obrigado.}
    \vfill
\end{frame}

\end{document}